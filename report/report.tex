\documentclass[10pt,twocolumn]{article}
\usepackage[utf8]{inputenc}
\usepackage{amsmath}
\usepackage{amsfonts}
\usepackage{amssymb}
\usepackage{graphicx}\usepackage[left=1in,right=1in,top=1in,bottom=1in]{geometry}

\title{\textbf{COMP90015 Project 2 \\ High Availability and Eventual Consistency\\}}

\author{\textbf{Thu Thao Le} (\texttt{thaol4@student.unimelb.edu.au})
\\[2ex] \textbf{Yicong Li} (\texttt{yicongl2@student.unimelb.edu.au})}

\date{}
\begin{document}
\maketitle

\section{Introduction}
The goal of this project is to implement new server protocol to address the server failure issues. The server can provide availability and eventual consistency among the servers and clients when network partioning happens. 

There are many challenges in this project. The hardest part is to rebuild the server protocol to handle the failures in the network. We have used tree protocol with some improvements.

The result:

\section{Server failure}
There are three properties in our distributed systems: Consistency, Availability and Network partitions tolerance. According to CAP theorem \cite{eric}, we can only achieve at most two out of three properties. Since there are failures in the servers and the connections, we can only achieve either consistency or availability. If we choose availability, we can always return the messages even it is not up-to-date data. If we choose consistency, we have to update the latest data before returning the messages to clients. That means we can delay the sending process until all data is up-to-date or just return error when we cannot update the data. 

In this project, we focus on the availability and the consistency can eventually be reached. 

\section{Server protocol}

\section{Availability}

\section{Eventual Consistency}

\section{Conclusion}

\begin{thebibliography}{9}
\bibitem{eric} 
Eric Brewer
\textit{Towards robust distributed systems}. Jan, 2000.
\end{thebibliography}

\end{document}